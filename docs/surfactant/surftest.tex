\documentclass[11pt]{article}

\textwidth 6.5in
\oddsidemargin=0in
\evensidemargin=0in
\textheight 9in
\topmargin -0.5in

\usepackage{graphicx,bm,amssymb,amsmath,amsthm}

\usepackage{showlabels}

\newcommand{\bi}{\begin{itemize}}
\newcommand{\ei}{\end{itemize}}
\newcommand{\ben}{\begin{enumerate}}
\newcommand{\een}{\end{enumerate}}
\newcommand{\be}{\begin{equation}}
\newcommand{\ee}{\end{equation}}
\newcommand{\bea}{\begin{eqnarray}} 
\newcommand{\eea}{\end{eqnarray}}
\newcommand{\ba}{\begin{align}} 
\newcommand{\ea}{\end{align}}
\newcommand{\bse}{\begin{subequations}} 
\newcommand{\ese}{\end{subequations}}
\newcommand{\bc}{\begin{center}}
\newcommand{\ec}{\end{center}}
\newcommand{\bfi}{\begin{figure}}
\newcommand{\efi}{\end{figure}}
\newcommand{\ca}[2]{\caption{#1 \label{#2}}}
\newcommand{\ig}[2]{\includegraphics[#1]{#2}}
\newcommand{\tbox}[1]{{\mbox{\tiny #1}}}
\newcommand{\mbf}[1]{{\mathbf #1}}
\newcommand{\half}{\mbox{\small $\frac{1}{2}$}}
\newcommand{\vt}[2]{\left[\begin{array}{r}#1\\#2\end{array}\right]} % 2-col-vec
\newcommand{\mt}[4]{\left[\begin{array}{rr}#1&#2\\#3&#4\end{array}\right]} % 2x2
\newcommand{\eps}{\varepsilon}
\newcommand{\bigO}{{\mathcal O}}
\newcommand{\sfrac}[2]{\mbox{\small $\frac{#1}{#2}$}}
\newcommand{\qqquad}{\qquad\qquad}
\newcommand{\R}{\mathbb{R}}
\newcommand{\C}{\mathbb{C}}
\newcommand{\N}{\mathbb{N}}
\newcommand{\Z}{\mathbb{Z}}
\DeclareMathOperator{\re}{Re}
\DeclareMathOperator{\im}{Im}
\DeclareMathOperator{\tr}{Tr}
\DeclareMathOperator{\res}{res}
\newtheorem{thm}{Theorem}
\newtheorem{lem}[thm]{Lemma}
\newtheorem{alg}[thm]{Algorithm}
\newtheorem{pro}[thm]{Proposition}
\newtheorem{cor}[thm]{Corollary}
\newtheorem{rmk}[thm]{Remark}
\newtheorem{exa}[thm]{Example}
\newtheorem{conj}[thm]{Conjecture}
% this work...
\newcommand{\x}{\mbf{x}}
\newcommand{\n}{\mbf{n}}
\newcommand{\kk}{\mbf{k}}
\newcommand{\uu}{\mbf{u}}
\newcommand{\vv}{\mbf{v}}



\begin{document}
\title{Testing a surface advection-reaction-diffusion solver on fixed or moving surfaces}
\author{Alex Barnett; with Yuan-Nan Young, Libin Lu, Mike Shelley}
\date{\today}
\maketitle
\begin{abstract}
  Notes on how to set up tests of correctness for a numerical solver
  for on-surface advection-diffusion equation of a scalar surface function,
  eg, a concentration.
  The surface is smooth, embedded in $\R^3$, and functions are smooth.
  Fixed surface and moving surfaces are discussed.
\end{abstract}

\section{Notes about calculus on and off surfaces}
\label{s:calc}

Let $\Gamma \subset \R^3$ be a closed simply-connected surface,
with outward unit normal vector $\n$.

There is some confusion about the definition of a surface divergence
div$_s$ of a
vector field, especially for the case of a nontangential one.
The notation $\nabla_s \cdot$ for div$_s$ is also poor since the
surface divergence is not simply a dot with the vector of partials,
but we use it.

For any vector field $\mbf{F}:\R^3\to\R^3$, not generally tangential,
\be
\nabla_s \cdot \mbf{F} = 
\nabla \cdot \mbf{F} - \n\cdot(\nabla \mbf{F}) \cdot \n
\label{divs}
\ee
in terms of the Euclidean divergence, and the Euclidean Jacobian tensor
$\nabla \mbf{F}$ with Cartesian elements $\partial_i F_j$.

A potential confusion is then how to define 
$\nabla_s \cdot \mbf{F}$ for a field defined only on-surface,
ie $\mbf{F}:\Gamma \to \R^3$, where $\nabla \mbf{F}$ is not available.
The resolution is that that some extension is needed, but the
answer is independent of how $\mbf{F}$
is extended off $\Gamma$; eg it may be extended as a constant vector
locally along normals, or extended in any other smooth way.


\begin{exa}
  For $\Gamma$ the unit sphere at the origin, one may extend the normal
  field $\n$ on $\Gamma$ by
  i) $\mbf{F} = \x$ which is nonconstant in the normal direction,
  or by ii) $\mbf{F} = \x/\|\x\|$ which is constant along normals
  (at least locally, since it is not true at the origin).
  For i) then $\nabla\cdot\x = 3$ while $\n\cdot\nabla\mbf{F}\cdot\n =
  \n\cdot I \cdot \n = 1$, so their difference is $\nabla_s\mbf{F} = 2$.
  Instead taking ii) at the point $\x=\n =(0,0,1)$,
$\nabla\cdot (\x/r) = 2$ where $r=\|\x\|$, while $\n\cdot\nabla\mbf{F}\cdot\n = 0$ because $F_z$ is constant wrt $z$.
  Both verify \eqref{divsn} below since $H$ is the inverse of a sphere's radius.
\end{exa}

A special case is the normal field on $\Gamma$, for which
\be
\nabla_s\cdot \n = 2H
\label{divsn}
\ee
where $H$ is the mean curvature, a scalar function on $\Gamma$,
with sign positive for a sphere with
outward normal. $2H$ is the {\em total curvature} (sometimes called
$\kappa$ in the 2D case where there is only one principal curvature,
but we avoid this symbol in 3D due to confusion with the Gaussian curvature $K$).
(Note re code: {\tt ves3d} has negative definition of $H$.)

The split of any $\mbf{F}:\Gamma \to \R^3$ into tangential and normal
parts is $\mbf{F} = \mbf{F}_s + \n\cdot\mbf{F}$ where
\be
\mbf{F}_s := (I-\n\n^T)\mbf{F} = \mbf{F} - (\n\cdot\mbf{F})\n
.
\label{proj}
\ee

Given any smooth vector field $\mbf{F}$ in $\R^3$, with tangential part
$\mbf{F}_s$ on $\Gamma$,
a useful formula (eg Marz--MacDonald 2012, ``Calculus on surfaces with general closest point functions'', \S2.3)
relating surface divergence to the surface divergence of the tangential part is
\be
\nabla_s \cdot \mbf{F}
\;=\;
\nabla_s \cdot \mbf{F}_s + 2H \n\cdot \mbf{F}
.
\label{divss}
\ee
Unlike \eqref{divs}, this provides a formula for surface divergence of
a nontangential vector field referring only to values on $\Gamma$.
%since $\nabla_s\cdot \mbf{F}_s$ may be written using
It is easy to prove by applying the product rule
$\nabla_s\cdot(a\vv) = a \nabla_s\cdot\vv + \vv \cdot \nabla_s a$,
where $\nabla_s := \nabla + \n (\n \cdot \nabla \cdot)$ is the tangential
gradient, to \eqref{proj} with $a = \n\cdot\mbf{F}$ and $\vv=\n$.
This product rule is in turn verified via \eqref{divs} by
an arbitrary smooth extension of $a$ and $\vv$ off $\Gamma$.

\begin{exa}
  Again for $\Gamma$ the unit sphere, if $\mbf{F}=\n$, then $\mbf{F}_s\equiv\mbf{0}$,
  so \eqref{divss} gives $\nabla_s\cdot\mbf{F} = 2H \equiv 1$, accounted
  for purely by the normal component.
\end{exa}
\begin{exa}
  Again for $\Gamma$ the unit sphere,
  $\mbf{F}=(0,0,1)$ has vanishing surface divergence. 
  This may be seen via a) constant extension to $\R^3$ giving
  both terms vanishing in \eqref{divs}, showing the surface divergence
  vanishes everywhere.
  Alternatively, b) without extension
  we use \eqref{divss} at the north pole, for instance.
  The two terms cancel: $\nabla_s\cdot \mbf{F}_s = -1$
  by local projection onto the chart $z(x,y) = \sqrt{1-x^2+y^2}$
  in which coordinates $\mbf{F}_s = (-x,-y) + \bigO((x,y)^3)$ and
  the metric $g = I + \bigO((x,y)^2)$.
  The second term $2H\n\cdot\mbf{F} = +1$.
  Thus the tangential part has negative surface divergence but this
  is cancelled by the curvature times normal component.
  \end{exa}


Note the differential geometry formula
$\nabla_s\cdot\mbf{F} = (\det g)^{-1/2} \partial_j (\sqrt{\det g} F^j)$
is not so useful for us, except in verifying the formula
at the north pole above, where $\det g(x,y) = 1/z = 1 + \bigO((x,y)^2)$.



\section{Fixed surface case}
\label{s:fixed}

For now fix $\Gamma$, so fluid velocity $\mbf{u}\equiv \mbf{0}$ on $\Gamma$,
and we do not have to worry about moving-surface terms.

In this case,
for all $\x\in\Gamma$ and all $t>0$,
we are given a diffusion rate function $D(\x,t)$,
a source function $f(\x,t)$,
and a tangential advection field $\vv_s(\x,t)$,
where the subscript $s$ is simply a reminder that
$\n\cdot\vv_s\equiv 0$.
In the static-surface case, our $\vv_s$ would be interpreted as the tangential
projection of the pulling force.

The general PDE IVP for a surface concentration $c(\x,t)$ is then
\bea
\partial_t c + \nabla_s \cdot(\vv_s c) &= &\nabla_s \cdot (D \nabla_s c) + f
, \qquad \x\in\Gamma, \;t>0
\label{pde}
\\
c(\x,0) & = & c_0(\x), \qqquad\qquad \x\in\Gamma
\label{ic}
\eea
where $c_0$ is the initial data.

\subsection{Material conservation test}

Setting $f\equiv0$, and for any surface transport
field $\vv_s$ and diffusion function $D$, then any solution to \eqref{pde},
the total amount of surface material
$$
C(t) := \int_\Gamma c \,dS
$$
is contant in time.
This would be easiest to test separately in the cases of advection
only then diffusion only.


\subsection{Reaction-diffusion only test}

An easy way to generate a constant-surface-diffusion test solution is
by restriction of a heat equation solution in $\R^3$.
Setting $D$ const in space and time and $\vv_s\equiv\mbf{0}$,
the surface PDE is
\be
\partial_t c \;=\;
D \Delta_s c + f
, \qquad \x\in\Gamma, \;t>0
\label{surfheat}
\ee
where $\Delta_s := \nabla_s\cdot (\nabla_s \cdot)$ is the Laplace--Beltrami
operator.
Letting $\Delta$ be the Laplacian in $\R^3$, a standard formula is
\be
\Delta \;=\; \Delta_s + \partial_{nn} + 2H \partial_n
\label{LB}
\ee
where the total curvature, a scalar function on $\Gamma$,
is given by \eqref{divsn}.
Obviously
$\partial_{n} u= \n \cdot \nabla u = \sum_{i=1}^3 n_i \partial_{x_i} u$.
Less obviously,
$\partial_{nn} u= \n\cdot (\nabla\nabla u) \cdot \n = \sum_{i,j=1}^3 n_i n_j \partial_{x_ix_j} u$, ie, there are no surprises because in a local surface-conforming $(\mbf{s},n)$
coordinate system we have chosen $n$ to correspond to signed distance in the
normal direction.

The test procedure for a solver for \eqref{surfheat} is:
\ben
\item
  Evaluate $2H$ on $\Gamma$ (meaning, at all surface nodes). This
  is purely geometry.
\item Pick an analytic
  solution $u$ to the heat equation in $\R^3$,
  \be
  u_t \;=\;D \Delta u,
  \label{heat}
  \ee
  eg a decaying
  plane wave with generic wavevector $\kk\in\R^3$ and phase offset $\phi\in\R$,
  $$
  u(\x,t) = e^{-D\|\kk\|^2 t} \sin(\kk\cdot \x + \phi).
  $$
  Increasing the magnitude of $\kk$ would test accuracy vs higher spatial
  resolution.
\item
  Set up a function to evaluate a source
  \be
  f \;=\; D(\partial_{nn} u + 2H \partial_n u)
  \label{f}
  \ee
  on $\Gamma$ for all times $t$. Use the analytic partials of $u$ in $\R^3$.
\item
  Apply the solver to \eqref{surfheat}, using the above $f$ source function,
  with initial conditions $u_0 = u(\cdot,0)|_\Gamma$,
  to generate a numerical $c(\x,t)$ on the surface for future times.
\item
  Combining \eqref{LB}, \eqref{heat} and \eqref{f} shows that
  $c \equiv u|_\Gamma$ is the exact solution to \eqref{surfheat} with the above
  IC.
  Thus, measure and report the error $c - u|_\Gamma$.
\een

\subsection{Advection-reaction only test}

This tests advection is correct on a fixed surface for $D\equiv 0$,
using the simplest advection in 3D.
Pick a constant 3D advection vector $\vv\in\R^3$.
Let
$$
\vv_s = \vv - (\n\cdot\vv)\n  \qquad \mbox{ on } \Gamma
$$
be its tangential part, which will advect material
from one end of the vesicle to the other, but slowing down at these ends (poles).

Let $u_0:\R^3\to\R$ be a fixed test function, eg a Gaussian
$u_0(\x) = e^{-\|x\|^2/2\sigma^2}$ with numerical support centered at the vesicle
and $\sigma$ about the size of the vesicle.
We construct an $f$ such that $c \equiv u|_\Gamma$ for the advecting
field
\be
u(\x,t) = u_0(\x - t\vv)
\label{u}
\ee

Set $\mbf{F} := \vv u$, and combining \eqref{divs} and \eqref{divss}
gives
\bea
\nabla_s\cdot(\vv_s u) &=& \nabla\cdot (\vv u) - \n\cdot (\nabla(\vv u)) \cdot n
- 2H \n \cdot\vv u
\nonumber\\
&=& \vv \cdot \nabla u - (\n\cdot\vv) \n\cdot \nabla u - 2H \n \cdot \vv u.
\eea
Since $\partial_u = -\vv\cdot u$ by \eqref{u}, the first term cancel
and the source
\be
f = \partial_u + \nabla_s\cdot(\vv_s u) = (\n\cdot\vv)(\n\cdot \nabla u - 2H u)
\label{fadv}
\ee
results in the restriction
$c(\x) = u_0(\x-t\vv)$ for $\x\in\Gamma$, $t>0$, being a
solution to \eqref{pde} with $D\equiv 0$.

For the numerical test,
evaluate \eqref{fadv} as the source, using
$\nabla u(\x,t) = \nabla u_0(\x-t\vv)$.
The difference between $c$ and $u$ on $\Gamma$ gives an error metric.

It may be possible to combine the last two tests via
a 3D solution of advection-diffusion (a translating heat solution).
It is probably not necessary.


\section{Deforming vesicle (moving surface) case}

Stone 1990 derives transport on a deforming surface.
\eqref{divs} and \eqref{divss} are used in going from his (5) to his (6),
ie, writing the nontangential surface divergence
of the flux $\mbf{F} = c\uu$ where $c$ is concentration and $\uu$ the
fluid velocity (hence local
velocity of $\Gamma$),
using on-surface values only, as
$$\nabla_s \cdot (c\uu) \;=\;
\nabla_s\cdot (c\uu_s) + (\nabla_s\cdot\n)
(\n\cdot\uu) c.$$

The mass conservation test will also apply here.

In testing advection of a scalar function $f$ on moving
$\Gamma(t)$, the case of a moving vesicle surface
may not be generic enough,
since: $dS$ locally conserved due to constant-area constraint,
so that $\nabla_s \cdot \uu \equiv 0$, so that the flux of $c$
is $\uu_s \cdot \nabla_s c$.
This meas that in a Lagrangian frame,
$(d/dt) \dot c  = 0$ for advection with $\uu$ and no diffusion term.

An expanding sphere would  be a good test where $dS$ is not conserved.
But it may not be needed if we always care about vesicles.

*** TO DO: add Laplace-Beltrami diffusion term to $c$ in Lagrangian frame.



%\bibliographystyle{abbrv}
%\bibliography{refs}

\end{document}
